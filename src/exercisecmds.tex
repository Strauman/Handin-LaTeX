
%!TEX root = main.tex
%\BDOC
%:§ex
%:Here are commands related to creating exercises
%:-
\makeatletter
\newlength{\partproblemheadermargin}
\newlength{\problemheadermargin}
\newlength{\probleminset}
\setlength{\problemheadermargin}{0em}
\setlength{\probleminset}{3em}
\setlength{\partproblemheadermargin}{2em}


\def\fullproblemmargin{\dimexpr -\extramargins/2+\problemheadermargin\relax}
\def\fullpartproblemmargin{\dimexpr\partproblemheadermargin\relax}


\newcommand{\@atMargin}[2]{\strut\vadjust{\@domark{#1}{#2}}}

\newcommand{\@domark}[2]{%
  \hbox to #2{
    \vbox to 0pt{
      \kern-\dp\strutbox
      \smash{\llap{#1}}
      \vss
    }%
  }
}
\gdef\isFirstProblem{1}
\def\AtFirstProblem{
\begin{addmargin}{\probleminset}
\preto\@enddocumenthook{\end{addmargin}}
}
%Macro containing exercise number
%!! DOCS ON BOTTOM OF FILE !!
\gdef\currentProblem{0}
\gdef\currentPartProblem{0}
% For backward compability
\def\exerciseNr{\currentProblem}
%:§ref
%:=\currentProblem
%:Contains the argument of the last call to \problem
%:!\pproblem{hello}
%:!\pproblem{world}
%:!\currentPartProblem %<- contains world
%:=\currentPartProblem
%: Just as \refCom{currentProblem}, but contains the argument of the last call to \pproblem
%:=\currentProblemIfNewPage
%:If the problem is not defined on the current page, then
%:The first time \currentProblemIfNewPage is called on a page,
%:it expands to the current problem number. If not, it expands to empty.
%:If the problem is defined on the current page, it also expands to empty.
%: This is used in the default key for the \keyRef{part problem header}.
%: Here are some examples
%:!\problem{1}
%:!\currentProblemIfNewPage % <- empty
%:
%:!\problem{4}
%:!\clearpage
%:!\currentProblemIfNewPage % <- expands to 4
%:!\currentProblemIfNewPage % <- expands to empty
%:-
% Functionality for showing e.g 1c) if it is the first part problem on
% the page and only e.g. d) if it is not
\global\let\@ProblemOnThisPage\@isTrue%
\AddEverypageHook{%
\global\let\@ProblemOnThisPage\@isFalse%
}
\gdef\currentProblemIfNewPage{\if\@ProblemOnThisPage\@isFalse%
\global\let\@ProblemOnThisPage\@isTrue%
\currentProblem\fi}%
%:§ex
%:=\problem{text}
%: This command will print out a problem header. For example \problem{1}
%: prints a nice big header \!textbf{Problem 1} You can do a star (*) after \problem to
%: prevent it from showing in the table of contents
%:-
%:=\problem*{text}
%: Does the same as \problem, but does not add the problem to the table of contents
%:-
\gdef\problem{\@ifstar{\@problem}{\@problemTOC}}
\gdef\@problemTOC#1{%
  \gdef\currentProblem{#1}%
  \addcontentsline{toc}{section}{\handin@opt@problemheaderTOC}%
  \@problem{#1}%
}
\def\@problem#1{%
\global\let\@ProblemOnThisPage\@isTrue
\def\isTrue{1}%
\if\isFirstProblem\isTrue%
\AtFirstProblem%
\gdef\isFirstProblem{0}%
\fi%
% Update exercise number
  \gdef\currentProblem{#1}%
  \vbox to 0pt{%
    \hspace*{\fullproblemmargin}{\normalfont\Large\bfseries\handin@opt@problemheader}%
  }\vspace*{1.5\baselineskip}%
}
%:=\pproblem{text}
%: This command will print out a part problem header based on what problem you are on.
%: For example if you already have done \problem{1}, then \pproblem{a}
%: prints a nice big header \!textbf{(1a)}. Note that the default behaviour is such that if you are on a
%: new page, then the part problems are shown with the exercise number in front of it. If not
%: it is omitted. If you want to change this behaviour, see \keyRef{part problem header}
%:-
%:=\pproblem*{text}
%: Does the same as \pproblem, but does not add the part problem to the table of contents
%:-
\gdef\pproblem{\@ifstar{\@pproblem}{\@pproblemTOC}}
\gdef\@pproblemTOC#1{%
  \gdef\currentPartProblem{#1}
  \addcontentsline{toc}{subsection}{\handin@opt@partproblemheaderTOC}%
  \@pproblem{\currentPartProblem}%
}
\def\@pproblem#1{%
% Write out problem number and letter
\@atMargin{{\normalfont\Large\bfseries\handin@opt@partproblemheader}}{\fullpartproblemmargin}\ignorespaces%
}

\makeatother
