%!TEX root = main.tex
\makeatletter
\edef\@isTrue{1}
\edef\@isFalse{0}
%:§config
%: You can do configurations on this package, and probably
%: even more to come in later versions!
%:-
\pgfkeys{
/handinsetup/.is family, /handinsetup,
default/.style = {
problem header={\@tr{Problem}~\currentProblem},
part problem header={\currentProblemIfNewPage\currentPartProblem)},
problem TOC={\handin@opt@problemheader},
part problem TOC={\currentPartProblem},
logo width={0.4},
title style=wholepage,
},
% Expanded macros
% Unexpanded macros
problem header/.store in = \handin@opt@problemheader,
part problem header/.store in = \handin@opt@partproblemheader,
problem TOC/.store in = \handin@opt@problemheaderTOC,
part problem TOC/.store in = \handin@opt@partproblemheaderTOC,
logo width/.estore in = \handin@opt@logowidth,
% Bools
% part problems/.style = {switches/#1/.get = \exbank@opt@partProblems},
% Dict lookup
title style/.style = {titles/#1/.get = \handin@opt@titlestyle},
titles/.cd,
  wholepage/.initial = \handin@title@wholepage,
  small/.initial = \handin@title@small,
% True/False
switches/.cd,
% True
  On/.initial = \@isTrue,
  on/.initial = \@isTrue,
  T/.initial =  \@isTrue
% False
  Off/.initial = \@isFalse,
  off/.initial = \@isFalse,
  F/.initial = \@isFalse,
}
%:=\handinsetup{[key/values]}
\newcommand\handinsetup[1]{
  \pgfkeys{/handinsetup, #1}%
}
\handinsetup{default}
%: Here is a list of the different keys and their meaning
%:-
%:\keyDef{problem header}{macro}{\@tr{Problem}~\currentProblem}{
%:This options contains the formatting of the problem header. Use \currentProblem to access the number of the current exercise, and \@tr{Problem} to access the translation of Problem.\\
%:Defaults to \@tr{Problem}~\currentProblem
%:}
%:\keyDef{part problem header}{macro}{See below}{
%: This defaults to: \currentProblemIfNewPage\currentPartProblem).
%: See \refCom{currentProblemIfNewPage} and \refCom{currentPartProblem}
%:}
%:\keyDef{problem TOC}{macro}{\meta{contents of problem header}}{
%: This key decides what is written to the table of contents. It defaults to extract the content in \refKey{problem header}
%:}
%:\keyDef{part problem TOC}{macro}{\meta{contents of part problem header}}{
%: This key decides what is written to the table of contents. It defaults to \currentPartProblem
%: (which is what the user last sent to \pproblem).
%:}
%:\keyDef{logo width}{number}{0.4}{
%: Give as factor (between 0 and 1) of total text width.
%: This is a temporary fix for logo not being customisable, and is scheduled to be updated in a later version.
%:}
%:\keyDef{title style}{wholepage or small}{wholepage}{
%:If "wholepage", \maketitle will produce a full front-page. If "small", \maketitle will produce a smaller title containing the course name, course title, title, and author.
%:}
%:-
%: \!subsubsection*{Example} If you don't want to have the exercise number ever in front of the letter, then you'd do
%:! \handinsetup{part problem header=\currentPartProblem}
%:-
